\chapter{Introducci'on}

\textit{Necesito una cita cool para empezar mi tesis.}
\begin{flushright}
 Fleo
 \end{flushright}

\vspace{10 pt}


%business dynamics
%sistemas complejos adaptativos
%poner ejemplos?

%dynamic stability : edge of chaos. no es que sean caóticos sino que la mayor parte de las fluctuaciones pequeñas se las comen los feedback, pero la línea de qué es "pequeño" no es nada clara
%ley de Ashby: para poder controlar algo, se necesita al menos igual nivel de complejidad


Uno de las principales dificultades de las cadenas de suministro es que los agentes encargados de optimizar las estrategias solamente pueden tomar decisiones "dentro" del eslabón en el que se encuentran, y no tienen información más allá de los eslabones inmediatamente conectados. Así, la información acerca de la demanda del consumidor se va diluyendo en cada nivel, además de que las decisiones tomadas tienen repercusiones más allá del futuro inmediato. \\

Los agentes optimizadores deben tratar de inferir el patrón global por medio de información local bastante restringida. Sin embargo, los datos que reciben obedecen al tiempo real y no tienen la oportunidad de repetir experimentos.\\

Un modelo computacional que se comporte suficientemente parecido al mundo real, en el que todos los demás eslabones tomen estrategias que también maximizarían sus beneficios podría dar una opción: el experimento es replicable tantas veces como sea necesario y cada eslabón puede conocer una estrategia óptima para una gran cantidad de demandas de consumidor posibles.\\

En este trabajo se modelará el Problema de Distribución de Cerveza, \textit{The Beer Distribution Game}, planteado por primera vez por el profesor \citet{Forrester}, en la Escuela de Administraci\'on y Direcci\'on de Empresas Sloan del MIT en los años 60, para ilustrar el concepto de \textit{efecto l\'atigo}. \\

