\prefacesection{Abstract}

Este trabajo de Tesis propone representar el conocido \textit{Juego de la Distribuci\'on de la Cerveza} como un modelo multiagente de cadena de suministro, a\~nadiendo una restricci\'on tal que se aproxime m\'as a un problema de la vida real: la producci\'on de materia prima es finita y solamente ocurre en un periodo espec\'ifico. El acercamiento elegido fue la aplicaci\'on de dos m\'etodos de aprendizaje reforzado: \textit{Policy Iteration} y \textit{Q-learning}; se encontr\'o que ambas t\'ecnicas encuentran estrategias para los jugadores que les reportan mayores utilidades que comprar d\'ia a d\'ia, bajo un esquema de recompensas y castigos. En este trabajo se conclute que el aprendizaje reforzado no solamente es eficiente, sino suficientemente flexible como para identificar cambios permanentes en la demanda y adaptar las consecuentes pol\'iticas de compra. Finalmente, se explora la posibilidad de que en subsecuentes trabajos, se a\~nadan estrategias diferentes o restricciones adicionales al \textit{Problema de la Distribuci\'on de la Cerveza} para obtener modelos m\'as fieles a las condiciones del mundo externo y, por lo tanto, utilizables en un contexto real.
