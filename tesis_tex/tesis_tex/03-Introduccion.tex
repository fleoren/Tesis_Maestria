\chapter{Introducci'on}

\textit{Work is the curse of the drinking classes.}
\begin{flushright}
 Oscar Wilde
 \end{flushright}

\vspace{10 pt}

\section{Objetivo y presentaci\'on del problema}

% Problem Statement\\

Una cadena de suministro, seg\'un \citet{Sterman}, es el conjunto de estructuras y y procesos que una organizaci\'on utiliza para proporcionarle un producto a un cliente. Este producto puede ser tangible (como un autom\'ovil) o intangible (como un servicio). Sin embargo, esto es una sobresimplificaci\'on, dado que cualquier cadena en la vida real tiene muchos m\'as factores. Para ser completamente realista, un sistema de simulaci\'on de tal cadena de suministro contemplar\'ia todas las variables presentes en su entorno: desde aquellas intr\'insecas al proceso como los costos de almacenamiento y los tiempos de espera para el cumplimiento de las \'ordenes, hasta externas como precio de bienes sustitutos y complementarios; o aquellas m\'as complejas, como din\'amicas entre distintos agentes tal que los consumidores cambien de distribuidor despu\'es de cierto n\'umero de \'ordenes no cumplidas.\\

Sin embargo, es imposible replicar todas las condiciones de un sistema tan complejo como la econom\'ia en el mundo real, e incluso es dif\'icil incorporar un alto n\'umero de variables al tratar de optimizar un problema de este calibre. Tambi\'en es posible que incluso despu\'es de plantear un sistema tal con una cantidad manejable de condiciones, este sea tan complejo o falto de linearidad, que no exista una soluci\'on anal\'itica y sea necesario utilizar herramientas m\'as sofisticadas.\\

Por esta y otras razones, los problemas econ\'omicos a resolver matem\'atica o program\'aticamente deben hacer ciertos supuestos acerca de las condiciones que los rigen. Para asegurar la aplicabilidad de las soluciones, es necesario asegurar un equilibrio entre apego a la realidad (m\'as variables, menos supuestos) y control de complejidad (m\'as supuestos, menos variables).\\

Tomemos como ejemplo el Juego de Distribuci\'on de la Cerveza, un famoso problema acad\'emico de cadena de suministro. En la forma est\'andar del problema (que ser\'a detallada m\'as adelante) com\'unmente se supone que uno de los agentes que forman parte de la cadena, la f\'abrica, tiene acceso a las materias primas para producir cerveza en cualquier momento del a\~no y controla el calendario de producci\'on (un ejemplo se puede encontrar en \citet{Sterman}). Sin embargo, los datos reales de producci\'on anual de cebada, nos muestran que este es un supuesto poco razonable.\\

Por otro lado, en dicha forma est\'andar tambi\'en se supone que la demanda del consumidor es constante (salvo un salto \'unico en el cual el consumidor comienza a requerir el doble del inventario, y mantiene esta nueva demanda por el resto del juego). De nuevo, los datos en el mundo real nos muestra que las personas en EUA beben mucha m\'as cerveza cerca y durante las festividades de fin de a\~no. En M\'exico parece pertinente tambi\'en considerar un pico de consumo el d\'ia de Independencia, y en una versi\'on a\'un m\'as fidedigna de la tendencia de demanda, por\'ian incluirse eventos deportivos. Esto de nuevo nos muestra que el supuesto est\'andar de demanda de consumidor es poco razonable.\\

En el presente trabajo, se pretende ayudar a responder una pregunta espec\'ifica que proviene de la brecha causada por tales supuestos, que la autora juzga poco razonables o coherentes con el mundo real.\\

La hip\'otesis con la que se trabajar\'a se puede delinear de la siguiente manera:\\

\textit{`Bajo este nuevo conjunto de supuestos, es posible encontrar las estrategias \'optimas para todos los agentes tomadores de decisiones en el Juego de la Distribuci\'on de Cerveza, por medio de algoritmos de aprendizaje reforzado, que produzcan resultados en un tiempo suficientemente veloz como para poder accionarlas.'}\\

\section{Metodolog\'ia y resultado esperado}

Dado que el objetivo del Juego de la Distribuci\'on de Cerveza para cada agente (menudeo, mayoreo, almac\'en regional, f\'abrica) es minimizar su costo, t\'ipicamente las soluciones se plantean como aquellas para un problema de optimizaci\'on (\citet{Sterman}). Sin embargo recientemente se ha explorado la soluci\'on por medio de distintos m\'etodos, desde algoritmos gen\'eticos (\citet{Strozzi}) hasta aprendizaje de m\'aquina reforzado (\citet{Chaharsooghi}). En el presente trabajo se explorar\'an dos algoritmos del segundo tipo: \textit{policy iteration} y \textit{Q-learning}. Adem\'as, se remover\'a el supuesto relacionado a la producci\'on constante en los campos, ya que de esta manera se obtiene un modelo m\'as realista respecto a los ciclos naturales de cosecha de las materias primas.

\section{Organizaci\'on de la tesis}

Este trabajo est\'a organizado en ocho cap\'itulos. El primero, del cual forma parte esta secci\'on, proporciona una vista general y la estructura del texto.

Las secciones dos y tres establecen el contexto necesario para el problema a resolver; mientras que la cuarta muestra acercamientos previos a la soluci\'on, dentro de los cuales ya existen propuestas con los algoritmos presenta en este trabajo. La quinta secci\'on contiene una idea nueva para mejorar tales algoritmos conviertiendo el problema en uno m\'as cercano a la realidad: removiendo el supuesto de producci\'on ilimitada en los campos.

La secci\'on seis contiene la estructura del producto de datos construido para guardar los resultados y las principales diferencias necesarias dada la forma diferente de la soluci\'on dependiendo del algoritmo. La secci\'on siete contiene los resultados de este trabajo y una comparaci\'on entre ellos.

Por \'ultimo, la octava secci\'on contiene las conclusiones y posibles preguntas para investigaciones futuras en esta l\'inea.